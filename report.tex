
\documentclass{article}
\usepackage{graphicx} % Required for inserting images
\usepackage{listings}
\usepackage{adjustbox}
\title{Microc report}
\author{Giulio Piva}
\date{November 2023}

\begin{document}

\maketitle


\section{Parser}
The following is the grammar of microc:
\begin{lstlisting}[language=C, basicstyle=\ttfamily\fontsize{8pt}{14pt}, keywordstyle=\color{blue}, commentstyle=\color{green}]
Program::= Topdecl* EOF

Topdecl::= Varlist; | Fundecl | Structdecl ";"

Varlist::= Typ Vardecl*

Vardecl::= Vardesc "=" Expr

Vardesc::= ID | "*" Vardesc | "(" Vardesc ")"
| Vardesc "[""]" | Vardesc "[" INT "]"

Fundecl::= Typ ID "(" ((genericparam ",")* genericparam)? ")" Block

Structdecl::= "struct" ID "{" (genericparam ";")* "}"

genericparam:: Typ Vardesc

Block::= "{" (Stmt | Vardecl ";" | Vardecl "=" Expr ";") "}"

Typ::= "int" | "char" | "void"
| "bool" | "float" | "struct" ID

Stmt::= "return" Expr ";" | Expr ";"
| Block | "while" "(" Expr ")" Stmt
| "for" "(" Expr? ";" Expr? ";" Expr? ")" Stmt
| "do" Stmt "while" "(" Expr ")" ";"
| "if" "(" Expr ")" Stmt "else" Stmt
| "if" "(" Expr ")" Stmt

Expr::= RExpr | LExpr

LExpr::= ID | "(" LExpr ")"
| "*" LExpr | "*" AExpr
| LExpr "[" Expr "]"
| LExpr "." ID

RExpr::= AExpr | "sizeof" "(" Expr ")"
| ID "(" ((Expr ",") Expr)? ")"
| LExpr "=" Expr| Lexpr Shortop Expr | "!" Expr
| "-" Expr | "~" Expr|Expr BinOp Expr| "++" Lexpr
| "--" Lexpr | Lexpr "++" | Lexpr "--"

Bin0p::= "+" | "-" | "*" | "%" | "/"
| "&&" | "||" | "<" |">""<=" | ">="
| "==" | "!=" | "&" | "|" | "^"| ">>" | "<<"

Shortop ::= "+=" | "-=" | "*=" | "/=" | "&="

AExpr ::= INT | CHAR | BOOL | FLOAT | STRING | "NULL"
| "(" RExpr ")" | "&" LExpr
\end{lstlisting}
It has been modified to accomodate the following features:
\begin{itemize}
    \item multiple variable declaration and initialization: multiple variables can be declared and initialized in a single statement
    like the following: \textbf{int a = 1, b = 2, c = 3;}. However, since function parameters and struct fields don't support initialization,
    two different rules have been created: varlist and genericparam. The first one is used for variable declarations,
    while the latter is used for declaring function parameters and struct fields.
    \item sizeof operator: sizeof is considered as operator and not as a function. As such, its name has been added to the list of reserved keywords.
    \item bitwise operators
    \item short assignment operators
    \item Struct declaration and access
    \item do-while loop
    \item String literals
    \item Floating point numbers
\end{itemize}
\subsection*{Grammar disambiguation}
\begin{itemize}
  \item dangling else: give precedence to if statement with an empty else branch
  \item Binary operators: inline the BinOp rule
  \item Bitwise operators
\end{itemize}

\section{Symbol Table}
The symbol table has been implemented as a list of hash tables to resemble a hierarchical scope structure.
Therefore, the head of the list represents the local scope, whereas the tail represents the global scope.
For convenience and efficiency, three different symbol tables have been created for storing information about variables, functions and structs.
Those three tables are then composed into a single object which is carried throughout the sematic analysis and code generation phases.
The type signature of a symbol list is the following:
\begin{lstlisting}
type 'a t = (string, 'a) Hashtbl.t list
\end{lstlisting}

The type signature of the context object is the following:
\begin{lstlisting}
type context = {
  fun_symbols : 'a Symbol_table.t;
  var_symbols : 'b Symbol_table.t;
  struct_symbols : 'c Symbol_table.t;
}
\end{lstlisting}

The type of context assumes different values during the semantic analysis and code generation phases, which will explained in the following sections.
\section{Semantic Analysis}
The aim of the semantic analysis of a microc program is to check the well-typedness of the program.
This phase recursively traverses the AST and checks that the statements of the programs satisfy the rules of the type system.
The relevant functions are the following:

\begin{itemize}
  \item \textbf{fundecl\_type\_check}
  \item \textbf{structdecl\_type\_check}
  \item \textbf{var\_type\_check}
  \item \textbf{stmt\_type\_check}
  \item \textbf{stmtordec\_type\_check}
\end{itemize}
All of these leverage the following other functions to obtain the type of their arguments and expressions:
\begin{itemize}
  \item \textbf{expr\_type}
  \item \textbf{access\_type}
  \item \textbf{unaryexp\_type}
  \item \textbf{binaryexp\_type}
\end{itemize}

\subsection{Semantic analysis Symbol Table}
The type signature of the symbol table for this phase is the following:

\begin{lstlisting}[basicstyle=\ttfamily\fontsize{6pt}{1em}]
type var_info = (code_pos * fun_decl) Symbol_table.t;
type fun_info = (code_pos * typ) Symbol_table.t;
type struct_info = (code_pos * struct_decl) Symbol_table.t;

type context = {
  fun_symbols : fun_info Symbol_table.t;
  var_symbols : var_info Symbol_table.t;
  struct_symbols : struct_info Symbol_table.t
}
\end{lstlisting}


\subsection{Additional rules}
\paragraph*{Type unification}
   There are cases in which in which types should not directly be tested to be equal, but rather they should be checked if compatible.
   A helper function \textbf{match\_types} is used to implement the two following type checking rules:
    \begin{itemize}
        \item Variable initialization and assignment: Pointer variables can be assigned the value NULL
        \item unsized arrays are admitted as type of function parameters, i.e. declarations in the form of a[] or a[][2][2].
    \end{itemize}

\paragraph*{Global variable initialization}
Global variables can be initialized only with compile-time defined constants.

\paragraph*{String initialization}
Strings are implemented as arrays of characters , but arrays cannot neither be initialized nor assigned in microc.
Therefore, a special case has been added when checking the declaration of an array of characters to allow string declarations.
The function which handles string initialization is \textbf{init\_string}.

\paragraph*{Independent-order declarations}
Functions and structs can be declared in any order and can be used before their declaration.
This entails a first scan of the program to add the signatures to the
symbol table. The bodies instead are checked in the subsequent scan.

\paragraph*{deadcode detection}
Different forms of deadcode exist. For this project, I implemented the detection of code after a return statement.
To implement this analysis, a boolean flag has been added to the return type of \textbf{stmt\_type\_check} and \textbf{stmtordec\_type\_check}
to signal whether the scan should continue or not, that is if a return statement has been found.
In case of instructions after a return statement, an exception is raised.

\paragraph*{Runtime support functions}


\section{Code Generation}
The code generation phase is responsible for generating the LLVM bitcode of the program.
As before, the code generation phase recursively traverses the AST and generates the bitcode for each statement, mapping
each statement to the corresponding LLVM operation.
The relevant functions are the following:
\begin{itemize}
  \item \textbf{codegen\_fundecl}: maps a function declaration to a LLVM function
  \item \textbf{codegen\_structdecl}:
  \item \textbf{codegen\_stmt}: maps a statement to its corresponding LLVM operation.
  In particular if,while and do-while statements are mapped according to
  the following templates: \\
  \textbf{if statement}: \\

\begin{minipage}{0.20\textwidth}
\begin{verbatim}
  if(cond) {
    ...
  }
  else {
    ...
  }
  <remaining code>
\end{verbatim}
\end{minipage}
\hspace{1cm} $\rightarrow$ \hspace{0.5cm}
\begin{minipage}{0.45\textwidth}
\begin{verbatim}
br <cond>, label %then, label %else
then:
  ; then body
else:
  ; else body
  br label %cont
cont:
  ; remaining code
\end{verbatim}
\end{minipage}


%while
\textbf{while statement}: \\

\begin{minipage}{0.20\textwidth}
\begin{verbatim}
  while(cond) {
    ...
  }
  <remaining code>
\end{verbatim}
\end{minipage}
\hspace{1cm} $\rightarrow$ \hspace{0.5cm}
\begin{minipage}{0.45\textwidth}
\begin{verbatim}
br label %test
test:
  ; necessary code to evaluate cond
  br <cond>, label %body, label %cont

body:
  ; while body
  br label %test

cont:
  ; remaining code

\end{verbatim}
\end{minipage}

%do-while
% \textbf{do-while statement}: \\

% \begin{minipage}{0.20\textwidth}
% \begin{verbatim}
%   while(cond) {
%     ...
%   }
%   <remaining code>
% \end{verbatim}
% \end{minipage}
% \hspace{1cm} $\rightarrow$ \hspace{0.5cm}
% \begin{minipage}{0.45\textwidth}
% \begin{verbatim}
% br label %test
% test:
%   ; necessary code to evaluate cond
%   br <cond>, label %body, label %cont

% body:
%   ; while body
%   br label %test

% cont:
%   ; remaining code

% \end{verbatim}
% \end{minipage}
  \item \textbf{codegen\_access}: maps each variable, pointer and array access to the associated address in memory
  \item \textbf{codegen\_un\_op}
  \item \textbf{codegen\_bin\_op}
\end{itemize}

\subsection*{Code generation Symbol Table}
\begin{lstlisting}
type context =
  { fun_symbols : L.llvalue Symbol_table.t
  ; var_symbols : L.llvalue Symbol_table.t
  ; struct_symbols : (L.lltype * string list) Symbol_table.t
  }
\end{lstlisting}

\subsection*{Struct and Function generation}
As for the semantic analysis, Struct are generated beforehand to allow for their use before their declaration.
However, this time the struct are fully generated in advance, and not only their signature. This is done because
when using a struct object to access a field, an address must exist in LLVM to access it.
For a function declaration instead, its signature generation is sufficient to allow for its call.

\subsection*{Array parameters}
An array parameter is converted to a pointer. This was done either to resemble the C language practice
and to support the unsized array as parameter.
However this mechanism introduces some additional checks when accessing an array in the code.
For instance, let's consider the following code:
\begin{lstlisting}[basicstyle=\ttfamily\fontsize{8pt}{14pt}, keywordstyle=\color{blue}, commentstyle=\color{green}]
int foo(int a[]) {
  int b[2];
  b = a[0];
  return b;
}
\end{lstlisting}
At llvm level, \textbf{a} will have a pointer as type, whereas \textbf{b} will array type.
Therefore, when accessing an array, the two cases must be handled accordingly.
\end{document}

\subsection*{Global variable initialization}
Since the initialization of global variables can be performed only with compile-time defined constants,
for this kind of variables I employed operations on constants. 